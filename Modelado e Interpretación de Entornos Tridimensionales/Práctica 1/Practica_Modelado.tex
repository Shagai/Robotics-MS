\documentclass[a4paper, fontsize=11pt]{scrartcl} % A4 paper and 11pt font 
\usepackage[a4paper,left=3cm,right=2cm,top=2.5cm,bottom=2.5cm]{geometry}

\usepackage[T1]{fontenc} % Use 8-bit encoding that has 256 glyphs
\usepackage{fourier} % Use the Adobe Utopia font for the document - comment this line to return to the LaTeX default
\usepackage[spanish]{babel} % Spanish language/hyphenation
\selectlanguage{spanish}
\usepackage[utf8]{inputenc}
\usepackage{amsmath,amsfonts,amsthm} % Math packages
\usepackage{graphicx} % The graphicx package
\usepackage{placeins}
\usepackage{caption}
\usepackage{subcaption}


\usepackage{listings} % Insert Scripts
\usepackage{color} %red, green, blue, yellow, cyan, magenta, black, white
\definecolor{mygreen}{RGB}{28,172,0} % color values Red, Green, Blue
\definecolor{mylilas}{RGB}{170,55,241}

\lstset{language=Matlab,%
	%basicstyle=\color{red},
	breaklines=true,%
	morekeywords={matlab2tikz},
	keywordstyle=\color{blue},%
	morekeywords=[2]{1}, keywordstyle=[2]{\color{black}},
	identifierstyle=\color{black},%
	stringstyle=\color{mylilas},
	commentstyle=\color{mygreen},%
	showstringspaces=false,%without this there will be a symbol in the places where there is a space
	numbers=left,%
	numberstyle={\tiny \color{black}},% size of the numbers
	numbersep=9pt, % this defines how far the numbers are from the text
	emph=[1]{for,end,break},emphstyle=[1]\color{red}, %some words to emphasise
	%emph=[2]{word1,word2}, emphstyle=[2]{style},    
}

\usepackage{sectsty} % Allows customizing section commands
%\allsectionsfont{\centering \normalfont\scshape} % Make all sections centered, the default font and small caps

\usepackage{fancyhdr} % Custom headers and footers
\pagestyle{fancyplain} % Makes all pages in the document conform to the custom headers and footers
\fancyhead{} % No page header - if you want one, create it in the same way as the footers below
\fancyfoot[L]{} % Empty left footer
\fancyfoot[C]{} % Empty center footer
\fancyfoot[R]{\thepage} % Page numbering for right footer
\renewcommand{\headrulewidth}{0pt} % Remove header underlines
\renewcommand{\footrulewidth}{0pt} % Remove footer underlines
\setlength{\headheight}{13.6pt} % Customize the height of the header

\numberwithin{equation}{section} % Number equations within sections (i.e. 1.1, 1.2, 2.1, 2.2 instead of 1, 2, 3, 4)
\numberwithin{figure}{section} % Number figures within sections (i.e. 1.1, 1.2, 2.1, 2.2 instead of 1, 2, 3, 4)
\numberwithin{table}{section} % Number tables within sections (i.e. 1.1, 1.2, 2.1, 2.2 instead of 1, 2, 3, 4)

%\setlength\parindent{0pt} % Removes all indentation from paragraphs - comment this line for an assignment with lots of text

\newenvironment{myalign}{\par\nobreak\large\noindent\align}{\endalign} %Altering fontsize in equations globally

%----------------------------------------------------------------------------------------
%	TITLE SECTION
%----------------------------------------------------------------------------------------

\newcommand{\horrule}[1]{\rule{\linewidth}{#1}} % Create horizontal rule command with 1 argument of height

\title{	
	\normalfont \normalsize 
	\textsc{Master en Automática y Robótica - UPM} \\ [25pt] % Your university, school and/or department name(s)
	\horrule{0.5pt} \\[0.4cm] % Thin top horizontal rule
	\huge Práctica de Modelado e Interpretación de Entornos Tridimensionales \\ % The assignment title
	\horrule{2pt} \\[0.5cm] % Thick bottom horizontal rule
}

\author{Jorge Camarero Vera - 07052} % Your name

\date{\normalsize\today} % Today's date or a custom date

\begin{document}
	\maketitle
	
	\section{Explicación de la tarea}
	
	El alumno debe de realizar un trabajo sobre alguno (o varios) de los siguientes puntos:
	\begin{itemize}
		\item Influencia en la calibración del número de imágenes empleadas.
		\item Influencia en la calibración del número de puntos empleados.
		\item Influencia en la calibración del nivel de ruido añadido.
		\item Influencia en el cálculo de la matriz fundamental del número de imágenes empleadas.
		\item Influencia en el cálculo de la matriz fundamental del número de puntos empleados.
		\item Influencia en el cálculo de la matriz fundamental del nivel de ruido añadido.
	\end{itemize}
	
	Para evaluar la bondad de los resultados se recomienda:
	\begin{itemize}
		\item En la calibración: errores en la reproyección y errores en la reconstrucción tridimensional.
		En líneas generales se puede decir que la calibración es correcta si la media de los errores de
		reproyección es menor de 1.0 píxeles, y si la media de los errores de reconstrucción es
		menor de 0.3 mm. 
		
		\item En el cálculo de la matriz fundamental. En líneas generales se puede decir que la estimación
		es correcta si la media de los errores de geometría epipolar es menor de 0.004 
	\end{itemize}
	
	\subsection{Desarrollo de la tarea}
	
	\subsubsection{Ruido 0 y Puntos Usados 100\%}
	
	\subsubsection*{Imágenes 1 a 10}
	
	Para \textbf{cámara 1} se obtienen una media de error de reproyección $\bar{E} = 24.097$ y $E_{max} = 107.44$.\\
	Para \textbf{cámara 2} se obtienen una media de error de reproyección $\bar{E} = 2.453$ y $E_{max} = 4.060$.\\
	
	La media de error de \textbf{distancia en la reconstrucción} $\bar{E} = 1.6914$ y $E_{max} = 6.6826$.\\
	La matriz fundamental:
	\[
	M=
	\begin{bmatrix}
	000,000000088&      000,000001191   &   -000,000310236   \\
	000,000001521&      000,000000081     & -000,015218744   \\
	-000,000453551 &     000,012482541   &   001,000000000  \\
	
	\end{bmatrix}
	\]
	
	Los errores medios en la \textbf{geometría epipolar} son $\bar{E} = 0.0044$ y $E_{max} = 0.0169$
	
	\subsubsection*{Imágenes 11 a 20}
	
	Para \textbf{cámara 1} se obtienen una media de error de reproyección $\bar{E} = 2.63$ y $E_{max} = 5.102$.\\
	Para \textbf{cámara 2} se obtienen una media de error de reproyección $\bar{E} = 1.503$ y $E_{max} = 2.86$.\\
	
	La media de error de \textbf{distancia en la reconstrucción} $\bar{E} = 0.2379$ y $E_{max} = 0.9829$.\\
	La matriz fundamental:
	\[
	M=
	\begin{bmatrix}
	0.000000100216544&	0.000000934240225&	-0.000276407706374 \\
	0.000001743888247&	-0.000000018101515&	-0.015148390599440 \\
	-0.000499410792322&	0.012427501856287&	1.000000000000000  \\
	
	\end{bmatrix}
	\]
	
	Los errores medios en la \textbf{geometría epipolar} son $\bar{E} = 0.0044$ y $E_{max} = 0.0137$
	
	\subsubsection*{Imágenes 21 a 30}
	
	Para \textbf{cámara 1} se obtienen una media de error de reproyección $\bar{E} = 10.686$ y $E_{max} = 37.101$.\\
	Para \textbf{cámara 2} se obtienen una media de error de reproyección $\bar{E} = 4.8050$ y $E_{max} = 6.6550$.\\
	
	La media de error de \textbf{distancia en la reconstrucción} $\bar{E} = 2.4995$ y $E_{max} = 10.7615$.\\
	La matriz fundamental:
	\[
	M=
	\begin{bmatrix}
	0.000000087789471&	0.000001200139563&	-0.000310850488067 \\
	0.000001489232039&	0.000000049016222&	-0.015342677999877 \\
	-0.000445268773931&	0.012626417295149&	1.000000000000000  \\
	
	\end{bmatrix}
	\]
	
	Los errores medios en la \textbf{geometría epipolar} son $\bar{E} = 0.0046$ y $E_{max} = 0.0213$
	
	\subsubsection*{Imágenes 31 a 40}
	
	Para \textbf{cámara 1} se obtienen una media de error de reproyección $\bar{E} = 0.9550$ y $E_{max} = 2.7030$.\\
	Para \textbf{cámara 2} se obtienen una media de error de reproyección $\bar{E} = 0.6450$ y $E_{max} = 1.4190$.\\
	
	La media de error de \textbf{distancia en la reconstrucción} $\bar{E} = 0.1794$ y $E_{max} = 0.7306$.\\
	La matriz fundamental:
	\[
	M=
	\begin{bmatrix}
	0.000000061371051&	0.000001255981409&	-0.000323147028645 \\
	0.000001511522663&	0.000000052885787&	-0.015194689034991 \\
	-0.000442451089967&	0.012462567451273&	1.000000000000000  \\
	
	\end{bmatrix}
	\]
	
	Los errores medios en la \textbf{geometría epipolar} son $\bar{E} = 0.0044$ y $E_{max} = 0.0232$.
	
	\subsubsection*{Imágenes 1 a 20}
	
	Para \textbf{cámara 1} se obtienen una media de error de reproyección $\bar{E} = 0.7020$ y $E_{max} = 1.4275$.\\
	Para \textbf{cámara 2} se obtienen una media de error de reproyección $\bar{E} = 2.2275$ y $E_{max} = 3.6985$.\\
	
	La media de error de \textbf{distancia en la reconstrucción} $\bar{E} = 0.1809$ y $E_{max} = 0.7318$.\\
	La matriz fundamental:
	\[
	M=
	\begin{bmatrix}
	0.000000062843093&	0.000001119811100&	-0.000280939168868 \\
	0.000001593985850&	0.000000044250715&	-0.015234165071434 \\
	-0.000466424093885&	0.012498294940741&	1.000000000000000  \\
	
	\end{bmatrix}
	\]
	
	Los errores medios en la \textbf{geometría epipolar} son $\bar{E} = 0.0044$ y $E_{max} = 0.0155$.
	
	\subsubsection*{Imágenes 1 a 30}
	
	Para \textbf{cámara 1} se obtienen una media de error de reproyección $\bar{E} = 0.7080$ y $E_{max} = 1.5730$.\\
	Para \textbf{cámara 2} se obtienen una media de error de reproyección $\bar{E} = 1.7510$ y $E_{max} = 3.1760$.\\
	
	La media de error de \textbf{distancia en la reconstrucción} $\bar{E} = 0.1863$ y $E_{max} = 0.7939$.\\
	La matriz fundamental:
	\[
	M=
	\begin{bmatrix}
	0.000000066200299&	0.000001170034617&	-0.000295081976154 \\
	0.000001542215208&	0.000000050914558&	-0.015294074419089 \\
	-0.000452615584867&	0.012561559097746&	1.000000000000000  \\
	
	\end{bmatrix}
	\]
	
	Los errores medios en la \textbf{geometría epipolar} son $\bar{E} =  0.0045$ y $E_{max} = 0.0174$
	
	\subsubsection*{Imágenes 1 a 40}
	
	Para \textbf{cámara 1} se obtienen una media de error de reproyección $\bar{E} = 0.7095$ y $E_{max} = 1.7018$.\\
	Para \textbf{cámara 2} se obtienen una media de error de reproyección $\bar{E} = 1.3253$ y $E_{max} = 2.5728$.\\
	
	La media de error de \textbf{distancia en la reconstrucción} $\bar{E} = 0.1752$ y $E_{max} = 0.7002$.\\
	La matriz fundamental:
	\[
	M=
	\begin{bmatrix}
	0.000000064283253&	0.000001186317526&	-0.000299979461735 \\
	0.000001539242984&	0.000000047822181&	-0.015279117622892 \\
	-0.000450952433138&	0.012546956329632&	1.000000000000000  \\
	
	\end{bmatrix}
	\]
	
	Los errores medios en la \textbf{geometría epipolar} son $\bar{E} =  0.0044$ y $E_{max} = 0.0189$
	
	%%%%%%%%%%%%%%%%%%%%%%%%%%%%	%%%%%%%%%%%%%%%%%%%%%%%%%%%%
		%%%%%%%%%%%%%%%%%%%%%%%%%%%%
			%%%%%%%%%%%%%%%%%%%%%%%%%%%%
				%%%%%%%%%%%%%%%%%%%%%%%%%%%%	%%%%%%%%%%%%%%%%%%%%%%%%%%%%
	
	\subsubsection{Ruido 0.25 y Puntos Usados 100\%}
	
	\subsubsection*{Imágenes 1 a 10}
	
	Para \textbf{cámara 1} se obtienen una media de error de reproyección $\bar{E} = 692.3590$ y $E_{max} = 1.3236e+03$.\\
	Para \textbf{cámara 2} se obtienen una media de error de reproyección $\bar{E} = 6.2840$ y $E_{max} = 10.2460$.\\
	
	La media de error de \textbf{distancia en la reconstrucción} $\bar{E} = 144.3235$ y $E_{max} = 1.4118e+03$.\\
	La matriz fundamental:
	\[
	M=
	\begin{bmatrix}
	0.000000117100411&	0.000001131345453&	-0.000311474725909 \\
	0.000001570755386&	0.000000061776028&	-0.015083966032154 \\
	-0.000475633563992&	0.012362135082127&	1.000000000000000  \\
	
	\end{bmatrix}
	\]
	
	Los errores medios en la \textbf{geometría epipolar} son $\bar{E} = 0.0044$ y $E_{max} = 0.0169$
	
	\subsubsection*{Imágenes 11 a 20}
	
	Para \textbf{cámara 1} se obtienen una media de error de reproyección $\bar{E} = 0.6950$ y $E_{max} = 1.3810$.\\
	Para \textbf{cámara 2} se obtienen una media de error de reproyección $\bar{E} = 1.0190$ y $E_{max} = 2.1920$.\\
	
	La media de error de \textbf{distancia en la reconstrucción} $\bar{E} = 0.1814$ y $E_{max} = 0.7591$.\\
	La matriz fundamental:
	\[
	M=
	\begin{bmatrix}
	0.000000156037939&	0.000000737183207&	-0.000271944275461 \\
	0.000001880151275&	-0.000000019488705&	-0.014930093145264 \\
	-0.000539453389517&	0.012211518431754&	1.000000000000000  \\
	
	\end{bmatrix}
	\]
	
	Los errores medios en la \textbf{geometría epipolar} son $\bar{E} = 0.0044$ y $E_{max} = 0.0140$
	
	\subsubsection*{Imágenes 21 a 30}
	
	Para \textbf{cámara 1} se obtienen una media de error de reproyección $\bar{E} = 63.0760$ y $E_{max} = 147.7760$.\\
	Para \textbf{cámara 2} se obtienen una media de error de reproyección $\bar{E} = 4.1010$ y $E_{max} = 5.8400$.\\
	
	La media de error de \textbf{distancia en la reconstrucción} $\bar{E} = 0.2386$ y $E_{max} = 1.0385$.\\
	La matriz fundamental:
	\[
	M=
	\begin{bmatrix}
	0.000000113578372&	0.000001154215798&	-0.000313439391424 \\
	0.000001516168994&	0.000000045595351&	-0.015280922424553 \\
	-0.000458868977324&	0.012572433751323&	1.000000000000000  \\
	
	\end{bmatrix}
	\]
	
	Los errores medios en la \textbf{geometría epipolar} son $\bar{E} = 0.0046$ y $E_{max} = 0.0213$
	
	\subsubsection*{Imágenes 31 a 40}
	
	Para \textbf{cámara 1} se obtienen una media de error de reproyección $\bar{E} = 0.9480$ y $E_{max} = 2.6800$.\\
	Para \textbf{cámara 2} se obtienen una media de error de reproyección $\bar{E} = 0.6620$ y $E_{max} = 1.4720$.\\
	
	La media de error de \textbf{distancia en la reconstrucción} $\bar{E} = 0.1714$ y $E_{max} = 0.6777$.\\
	La matriz fundamental:
	\[
	M=
	\begin{bmatrix}
	0.000000088318099&	0.000001234622560&	-0.000336274922134 \\
	0.000001523915589&	0.000000055975462&	-0.015056739555185 \\
	-0.000453725230911&	0.012334077052558&	1.000000000000000  \\
	
	\end{bmatrix}
	\]
	
	Los errores medios en la \textbf{geometría epipolar} son $\bar{E} = 0.0044$ y $E_{max} = 0.0233$
	
	\subsubsection*{Imágenes 1 a 20}
	
	Para \textbf{cámara 1} se obtienen una media de error de reproyección $\bar{E} = 0.7050$ y $E_{max} = 1.4340$.\\
	Para \textbf{cámara 2} se obtienen una media de error de reproyección $\bar{E} = 0.9570$ y $E_{max} = 2.2500$.\\
	
	La media de error de \textbf{distancia en la reconstrucción} $\bar{E} = 0.2123$ y $E_{max} = 0.8775$.\\
	La matriz fundamental:
	\[
	M=
	\begin{bmatrix}
	0.000000074154279&	0.000001053814159&	-0.000268343869003 \\
	0.000001634939896&	0.000000037960901&	-0.015176285665707 \\
	-0.000482648365434&	0.012446065169930&	1.000000000000000  \\
	
	\end{bmatrix}
	\]
	
	Los errores medios en la \textbf{geometría epipolar} son $\bar{E} = 0.0044$ y $E_{max} = 0.0155$
	
	\subsubsection*{Imágenes 1 a 30}
	
	Para \textbf{cámara 1} se obtienen una media de error de reproyección $\bar{E} = 0.7103$ y $E_{max} = 1.5730$.\\
	Para \textbf{cámara 2} se obtienen una media de error de reproyección $\bar{E} = 1.7220$ y $E_{max} = 3.2167$.\\
	
	La media de error de \textbf{distancia en la reconstrucción} $\bar{E} = 0.1773$ y $E_{max} = 0.7187$.\\
	La matriz fundamental:
	\[
	M=
	\begin{bmatrix}
	0.000000085303056&	0.000001104720434&	-0.000289728344770 \\
	0.000001596688824&	0.000000041410448&	-0.015184813555536 \\
	-0.000473319174385&	0.012458742683712&	1.000000000000000  \\
	
	\end{bmatrix}
	\]
	
	Los errores medios en la \textbf{geometría epipolar} son $\bar{E} =  0.0045$ y $E_{max} = 0.0175$
	
	\subsubsection*{Imágenes 1 a 40}
	
	Para \textbf{cámara 1} se obtienen una media de error de reproyección $\bar{E} = 0.7495$ y $E_{max} = 1.8925$.\\
	Para \textbf{cámara 2} se obtienen una media de error de reproyección $\bar{E} = 1.5320$ y $E_{max} = 2.7728$.\\
	
	La media de error de \textbf{distancia en la reconstrucción} $\bar{E} = 0.1955$ y $E_{max} = 0.8283$.\\
	La matriz fundamental:
	\[
	M=
	\begin{bmatrix}
	0.000000087961941&	0.000001146576265&	-0.000302760417852 \\
	0.000001577040166&	0.000000036766354&	-0.015174367934413 \\
	-0.000468781958101&	0.012451759796459&	1.000000000000000  \\
	
	\end{bmatrix}
	\]
	
	Los errores medios en la \textbf{geometría epipolar} son $\bar{E} =  0.0045$ y $E_{max} = 0.0189$
	
	%%%%%%%%%%%%%%%%%%%%%%%%%%%%	%%%%%%%%%%%%%%%%%%%%%%%%%%%%
	%%%%%%%%%%%%%%%%%%%%%%%%%%%%
	%%%%%%%%%%%%%%%%%%%%%%%%%%%%
	%%%%%%%%%%%%%%%%%%%%%%%%%%%%	%%%%%%%%%%%%%%%%%%%%%%%%%%%%
	
	\subsubsection{Ruido 0.5 y Puntos Usados 100\%}
	
	\subsubsection*{Imágenes 1 a 10}
	
	Para \textbf{cámara 1} se obtienen una media de error de reproyección $\bar{E} = 0.7180$ y $E_{max} = 1.3910$.\\
	Para \textbf{cámara 2} se obtienen una media de error de reproyección $\bar{E} = 1.8010$ y $E_{max} = 3.3910$.\\
	
	La media de error de \textbf{distancia en la reconstrucción} $\bar{E} = 0.2182$ y $E_{max} = 0.8469$.\\
	La matriz fundamental:
	\[
	M=
	\begin{bmatrix}
	0.000000209866101&	0.000000923002216&	-0.000299344613809 \\
	0.000001726406393&	0.000000041394304&	-0.014589810905878 \\
	-0.000561159256193&	0.011911693192366&	1.000000000000000  \\
	
	\end{bmatrix}
	\]
	
	Los errores medios en la \textbf{geometría epipolar} son $\bar{E} = 0.0047$ y $E_{max} = 0.0176$
	
	\subsubsection*{Imágenes 11 a 20}
	
	Para \textbf{cámara 1} se obtienen una media de error de reproyección $\bar{E} = 0.7010$ y $E_{max} = 1.4340$.\\
	Para \textbf{cámara 2} se obtienen una media de error de reproyección $\bar{E} = 0.7460$ y $E_{max} = 1.7010$.\\
	
	La media de error de \textbf{distancia en la reconstrucción} $\bar{E} = 0.1819$ y $E_{max} = 0.7638$.\\
	La matriz fundamental:
	\[
	M=
	\begin{bmatrix}
	0.000000263829732&	0.000000235079095&	-0.000229489167640 \\
	0.000002254182543&	-0.000000045522304&	-0.014399859771080 \\
	-0.000651028277702&	0.011688000264035&	1.000000000000000  \\
	
	\end{bmatrix}
	\]
	
	Los errores medios en la \textbf{geometría epipolar} son $\bar{E} = 0.0045$ y $E_{max} = 0.0157$
	
	\subsubsection*{Imágenes 21 a 30}
	
	Para \textbf{cámara 1} se obtienen una media de error de reproyección $\bar{E} = 0.7230$ y $E_{max} = 1.8540$.\\
	Para \textbf{cámara 2} se obtienen una media de error de reproyección $\bar{E} = 2.9750$ y $E_{max} = 4.1790$.\\
	
	La media de error de \textbf{distancia en la reconstrucción} $\bar{E} = 0.4202$ y $E_{max} = 1.5255$.\\
	La matriz fundamental:
	\[
	M=
	\begin{bmatrix}
	0.000000186870212&	0.000001039924695&	-0.000325742773579 \\
	0.000001601121496&	0.000000026536361&	-0.015012270579635 \\
	-0.000503020414056&	0.012330531296000&	1.000000000000000  \\
	
	\end{bmatrix}
	\]
	
	Los errores medios en la \textbf{geometría epipolar} son $\bar{E} = 0.0048$ y $E_{max} = 0.0211$
	
	\subsubsection*{Imágenes 31 a 40}
	
	Para \textbf{cámara 1} se obtienen una media de error de reproyección $\bar{E} = 2.0640$ y $E_{max} = 4.8980$.\\
	Para \textbf{cámara 2} se obtienen una media de error de reproyección $\bar{E} = 0.6640$ y $E_{max} = 1.5040$.\\
	
	La media de error de \textbf{distancia en la reconstrucción} $\bar{E} = 0.1717$ y $E_{max} = 0.6780$.\\
	La matriz fundamental:
	\[
	M=
	\begin{bmatrix}
	0.000000158802512&	0.000001138609592&	-0.000353432296349 \\
	0.000001599633275&	0.000000041839533&	-0.014705787359326 \\
	-0.000498549716545&	0.012012810108360&	1.000000000000000  \\
	
	\end{bmatrix}
	\]
	
	Los errores medios en la \textbf{geometría epipolar} son $\bar{E} = 0.0046$ y $E_{max} = 0.0241$
	
	\subsubsection*{Imágenes 1 a 20}
	
	Para \textbf{cámara 1} se obtienen una media de error de reproyección $\bar{E} = 0.7330$ y $E_{max} = 1.6150$.\\
	Para \textbf{cámara 2} se obtienen una media de error de reproyección $\bar{E} = 0.7560$ y $E_{max} = 1.9585$.\\
	
	La media de error de \textbf{distancia en la reconstrucción} $\bar{E} = 0.1943$ y $E_{max} = 0.8230$.\\
	La matriz fundamental:
	\[
	M=
	\begin{bmatrix}
	0.000000119206951&	0.000000852796074&	-0.000240578362668 \\
	0.000001788026494&	0.000000009316504&	-0.014906586071003 \\
	-0.000540806203155&	0.012194876410826&	1.000000000000000  \\
	
	\end{bmatrix}
	\]
	
	Los errores medios en la \textbf{geometría epipolar} son $\bar{E} = 0.0045$ y $E_{max} = 0.0156$
	
	\subsubsection*{Imágenes 1 a 30}
	
	Para \textbf{cámara 1} se obtienen una media de error de reproyección $\bar{E} = 5.4980$ y $E_{max} = 23.0103$.\\
	Para \textbf{cámara 2} se obtienen una media de error de reproyección $\bar{E} = 1.7087$ y $E_{max} = 3.2683$.\\
	
	La media de error de \textbf{distancia en la reconstrucción} $\bar{E} = 0.1856$ y $E_{max} = 0.7892$.\\
	La matriz fundamental:
	\[
	M=
	\begin{bmatrix}
	0.000000138067418&	0.000000949424455&	-0.000281111455947 \\
	0.000001727311256&	0.000000015310687&	-0.014895475029528 \\
	-0.000525419198014&	0.012188694688137&	1.000000000000000  \\
	
	\end{bmatrix}
	\]
	
	Los errores medios en la \textbf{geometría epipolar} son $\bar{E} =  0.0046$ y $E_{max} = 0.0177$
	
	\subsubsection*{Imágenes 1 a 40}
	
	Para \textbf{cámara 1} se obtienen una media de error de reproyección $\bar{E} = 0.7147$ y $E_{max} = 1.7103$.\\
	Para \textbf{cámara 2} se obtienen una media de error de reproyección $\bar{E} = 1.6202$ y $E_{max} = 3.0133$.\\
	
	La media de error de \textbf{distancia en la reconstrucción} $\bar{E} = 0.1713$ y $E_{max} = 0.6819$.\\
	La matriz fundamental:
	\[
	M=
	\begin{bmatrix}
	0.000000146559230&	0.000001024957014&	-0.000304576348115 \\
	0.000001683886539&	0.000000010928766&	-0.014886348757960 \\
	-0.000517017382222&	0.012186954974592&	1.000000000000000  \\
	
	\end{bmatrix}
	\]
	
	Los errores medios en la \textbf{geometría epipolar} son $\bar{E} =  0.0046$ y $E_{max} = 0.0191$
	
	%%%%%%%%%%%%%%%%%%%%%%%%%%%%	%%%%%%%%%%%%%%%%%%%%%%%%%%%%
	%%%%%%%%%%%%%%%%%%%%%%%%%%%%
	%%%%%%%%%%%%%%%%%%%%%%%%%%%%
	%%%%%%%%%%%%%%%%%%%%%%%%%%%%	%%%%%%%%%%%%%%%%%%%%%%%%%%%%
	
	\subsubsection{Ruido 1 y Puntos Usados 100\%}
	
	\subsubsection*{Imágenes 1 a 10}
	
	Para \textbf{cámara 1} se obtienen una media de error de reproyección $\bar{E} = 0.7230$ y $E_{max} = 1.5130$.\\
	Para \textbf{cámara 2} se obtienen una media de error de reproyección $\bar{E} = 0.6690$ y $E_{max} = 1.7700$.\\
	
	La media de error de \textbf{distancia en la reconstrucción} $\bar{E} = 0.1884$ y $E_{max} = 0.7024$.\\
	La matriz fundamental:
	\[
	M=
	\begin{bmatrix}
	0.000000533916436&	0.000000350370732&	-0.000337314123774 \\
	0.000002144032492&	-0.000000088315263&	-0.013259722121398 \\
	-0.000778088363361&	0.010729135051339&	1.000000000000000  \\
	
	\end{bmatrix}
	\]
	
	Los errores medios en la \textbf{geometría epipolar} son $\bar{E} = 0.0060$ y $E_{max} = 0.0217$
	
	\subsubsection*{Imágenes 11 a 20}
	
	Para \textbf{cámara 1} se obtienen una media de error de reproyección $\bar{E} = 3.3360$ y $E_{max} = 8.5730$.\\
	Para \textbf{cámara 2} se obtienen una media de error de reproyección $\bar{E} = 0.6710$ y $E_{max} = 1.5060$.\\
	
	La media de error de \textbf{distancia en la reconstrucción} $\bar{E} = 0.3303$ y $E_{max} = 1.3164$.\\
	La matriz fundamental:
	\[
	M=
	\begin{bmatrix}
	0.000000561240628&	-0.000001294759565&	-0.000071761050750 \\
	0.000003417118365&	-0.000000144009031&	-0.012804637614335 \\
	-0.000999835349487&	0.010114109077134&	1.000000000000000  \\
	
	\end{bmatrix}
	\]
	
	Los errores medios en la \textbf{geometría epipolar} son $\bar{E} = 0.0060$ y $E_{max} = 0.0256$
	
	\subsubsection*{Imágenes 21 a 30}
	
	Para \textbf{cámara 1} se obtienen una media de error de reproyección $\bar{E} = 0.7350$ y $E_{max} = 1.9050$.\\
	Para \textbf{cámara 2} se obtienen una media de error de reproyección $\bar{E} = 0.6800$ y $E_{max} = 1.5790$.\\
	
	La media de error de \textbf{distancia en la reconstrucción} $\bar{E} = 0.2100$ y $E_{max} = 2.2277$.\\
	La matriz fundamental:
	\[
	M=
	\begin{bmatrix}
	0.000000448443810&	0.000000647803219&	-0.000374468206555 \\
	0.000001911924704&	-0.000000050064983&	-0.013965227482305 \\
	-0.000665686065379&	0.011383272510023&	1.000000000000000  \\
	
	\end{bmatrix}
	\]
	
	Los errores medios en la \textbf{geometría epipolar} son $\bar{E} = 0.0060$ y $E_{max} = 0.0238$
	
	\subsubsection*{Imágenes 31 a 40}
	
	Para \textbf{cámara 1} se obtienen una media de error de reproyección $\bar{E} = 0.7230$ y $E_{max} = 1.9000$.\\
	Para \textbf{cámara 2} se obtienen una media de error de reproyección $\bar{E} = 0.7210$ y $E_{max} = 1.6940$.\\
	
	La media de error de \textbf{distancia en la reconstrucción} $\bar{E} = 0.1768$ y $E_{max} = 0.6891$.\\
	La matriz fundamental:
	\[
	M=
	\begin{bmatrix}
	0.000000398760626&	0.000000771661055&	-0.000394750860373 \\
	0.000001900664997&	-0.000000028526373&	-0.013520192835996 \\
	-0.000666478627893&	0.010933283522721&	1.000000000000000  \\
	
	\end{bmatrix}
	\]
	
	Los errores medios en la \textbf{geometría epipolar} son $\bar{E} = 0.0057$ y $E_{max} = 0.0295$
	
	\subsubsection*{Imágenes 1 a 20}
	
	Para \textbf{cámara 1} se obtienen una media de error de reproyección $\bar{E} = 1.0710$ y $E_{max} = 2.8415$.\\
	Para \textbf{cámara 2} se obtienen una media de error de reproyección $\bar{E} = 2.4910$ y $E_{max} = 4.1555$.\\
	
	La media de error de \textbf{distancia en la reconstrucción} $\bar{E} = 0.1687$ y $E_{max} = 0.6561$.\\
	La matriz fundamental:
	\[
	M=
	\begin{bmatrix}
	0.000000329933650&	0.000000128947457&	-0.000186219403076 \\
	0.000002401438721&	-0.000000122882585&	-0.013700054433510 \\
	-0.000767433412988&	0.011061688328923&	1.000000000000000  \\
	
	\end{bmatrix}
	\]
	
	Los errores medios en la \textbf{geometría epipolar} son $\bar{E} = 0.0058$ y $E_{max} = 0.0200$
	
	\subsubsection*{Imágenes 1 a 30}
	
	Para \textbf{cámara 1} se obtienen una media de error de reproyección $\bar{E} = 1.0813$ y $E_{max} = 2.7373$.\\
	Para \textbf{cámara 2} se obtienen una media de error de reproyección $\bar{E} = 1.7077$ y $E_{max} = 3.0300$.\\
	
	La media de error de \textbf{distancia en la reconstrucción} $\bar{E} = 0.9948$ y $E_{max} = 3.3221$.\\
	La matriz fundamental:
	\[
	M=
	\begin{bmatrix}
	0.000000303306232&	0.000000556346998&	-0.000282613550017 \\
	0.000002079523099&	-0.000000045378961&	-0.013891562638955 \\
	-0.000678116296374&	0.011247043356539&	1.000000000000000  \\
	
	\end{bmatrix}
	\]
	
	Los errores medios en la \textbf{geometría epipolar} son $\bar{E} =  0.0057$ y $E_{max} = 0.0217$
	
	\subsubsection*{Imágenes 1 a 40}
	
	Para \textbf{cámara 1} se obtienen una media de error de reproyección $\bar{E} = 1.2607$ y $E_{max} = 3.4365$.\\
	Para \textbf{cámara 2} se obtienen una media de error de reproyección $\bar{E} = 1.0050$ y $E_{max} = 1.0050$.\\
	
	La media de error de \textbf{distancia en la reconstrucción} $\bar{E} = 0.1776$ y $E_{max} = 0.7222$.\\
	La matriz fundamental:
	\[
	M=
	\begin{bmatrix}
	0.000000321154220&	0.000000569167327&	-0.000288159055198 \\
	0.000002056804382&	-0.000000022269124&	-0.013848056598527 \\
	-0.000678278524661&	0.011198204531785&	1.000000000000000  \\
	
	\end{bmatrix}
	\]
	
	Los errores medios en la \textbf{geometría epipolar} son $\bar{E} = 0.0057$ y $E_{max} = 0.0228$
	
	%%%%%%%%%%%%%%%%%%%%%%%%%%%%	%%%%%%%%%%%%%%%%%%%%%%%%%%%%
	%%%%%%%%%%%%%%%%%%%%%%%%%%%%
	%%%%%%%%%%%%%%%%%%%%%%%%%%%%
	%%%%%%%%%%%%%%%%%%%%%%%%%%%%	%%%%%%%%%%%%%%%%%%%%%%%%%%%%
	
	\subsubsection{Ruido 0 y Puntos Usados 75\%}
	
	\subsubsection*{Imágenes 1 a 10}
	
	Para \textbf{cámara 1} se obtienen una media de error de reproyección $\bar{E} = 0.7460$ y $E_{max} = 1.5830$.\\
	Para \textbf{cámara 2} se obtienen una media de error de reproyección $\bar{E} = 0.6290$ y $E_{max} = 1.6800$.\\
	
	La media de error de \textbf{distancia en la reconstrucción} $\bar{E} = 0.1630$ y $E_{max} = 0.5962$.\\
	La matriz fundamental:
	\[
	M=
	\begin{bmatrix}
	0.000000100139894&	0.000001231096506&	-0.000330313001941 \\
	0.000001506778451&	0.000000090203864&	-0.015164781028763 \\
	-0.000450952163281&	0.012426989104967&	1.000000000000000  \\
	
	\end{bmatrix}
	\]
	
	Los errores medios en la \textbf{geometría epipolar} son $\bar{E} = 0.0044$ y $E_{max} = 0.0163$
	
	\subsubsection*{Imágenes 11 a 20}
	
	Para \textbf{cámara 1} se obtienen una media de error de reproyección $\bar{E} = 1.2170$ y $E_{max} = 2.8590$.\\
	Para \textbf{cámara 2} se obtienen una media de error de reproyección $\bar{E} = 0.6770$ y $E_{max} = 1.4260$.\\
	
	La media de error de \textbf{distancia en la reconstrucción} $\bar{E} = 0.2045$ y $E_{max} = 0.7923$.\\
	La matriz fundamental:
	\[
	M=
	\begin{bmatrix}
	0.000000108442423&	0.000001026715559&	-0.000305092871250 \\
	0.000001671738861&	-0.000000014706234&	-0.015196961303058 \\
	-0.000480186458499&	0.012481336948998&	1.000000000000000  \\
	
	\end{bmatrix}
	\]
	
	Los errores medios en la \textbf{geometría epipolar} son $\bar{E} = 0.0043$ y $E_{max} = 0.0138$
	
	\subsubsection*{Imágenes 21 a 30}
	
	Para \textbf{cámara 1} se obtienen una media de error de reproyección $\bar{E} = 4.6950$ y $E_{max} = 6.5500$.\\
	Para \textbf{cámara 2} se obtienen una media de error de reproyección $\bar{E} = 11.0050$ y $E_{max} = 13.9910$.\\
	
	La media de error de \textbf{distancia en la reconstrucción} $\bar{E} = 0.5052$ y $E_{max} = 1.8035$.\\
	La matriz fundamental:
	\[
	M=
	\begin{bmatrix}
	0.000000089055322&	0.000001163376643&	-0.000303441308762 \\
	0.000001535495267&	0.000000024400855&	-0.015308962923432 \\
	-0.000457429836414&	0.012597198055918&	1.000000000000000  \\
	
	\end{bmatrix}
	\]
	
	Los errores medios en la \textbf{geometría epipolar} son $\bar{E} = 0.0045$ y $E_{max} = 0.0207$
	
	\subsubsection*{Imágenes 31 a 40}
	
	Para \textbf{cámara 1} se obtienen una media de error de reproyección $\bar{E} = 0.6800$ y $E_{max} = 1.7180$.\\
	Para \textbf{cámara 2} se obtienen una media de error de reproyección $\bar{E} = 0.6130$ y $E_{max} = 1.4130$.\\
	
	La media de error de \textbf{distancia en la reconstrucción} $\bar{E} = 0.1628$ y $E_{max} = 0.6095$.\\
	La matriz fundamental:
	\[
	M=
	\begin{bmatrix}
	0.000000054796494&	0.000001231681516&	-0.000314808004933 \\
	0.000001536454159&	0.000000046684706&	-0.015207321418559 \\
	-0.000445657294596&	0.012471619607605&	1.000000000000000 \\
	
	\end{bmatrix}
	\]
	
	Los errores medios en la \textbf{geometría epipolar} son $\bar{E} = 0.0042$ y $E_{max} = 0.0170$
	
	\subsubsection*{Imágenes 1 a 20}
	
	Para \textbf{cámara 1} se obtienen una media de error de reproyección $\bar{E} = 0.8115$ y $E_{max} = 1.7010$.\\
	Para \textbf{cámara 2} se obtienen una media de error de reproyección $\bar{E} = 1.1465$ y $E_{max} = 2.3660$.\\
	
	La media de error de \textbf{distancia en la reconstrucción} $\bar{E} = 0.1685$ y $E_{max} = 0.6792$.\\
	La matriz fundamental:
	\[
	M=
	\begin{bmatrix}
	0.000000079069001&	0.000001169286103&	-0.000305721111935 \\
	0.000001559465708&	0.000000054589742&	-0.015213049666455 \\
	-0.000458297515265&	0.012478289616414&	1.000000000000000  \\
	
	\end{bmatrix}
	\]
	
	Los errores medios en la \textbf{geometría epipolar} son $\bar{E} = 0.0044$ y $E_{max} = 0.0154$
	
	\subsubsection*{Imágenes 1 a 30}
	
	Para \textbf{cámara 1} se obtienen una media de error de reproyección $\bar{E} = 0.8740$ y $E_{max} = 2.3663$.\\
	Para \textbf{cámara 2} se obtienen una media de error de reproyección $\bar{E} = 0.6500$ y $E_{max} = 1.7230$.\\
	
	La media de error de \textbf{distancia en la reconstrucción} $\bar{E} = 0.1747$ y $E_{max} = 0.6698$.\\
	La matriz fundamental:
	\[
	M=
	\begin{bmatrix}
	0.000000079020761&	0.000001177229073&	-0.000305854432101 \\
	0.000001544992419&	0.000000046922109&	-0.015253786017275 \\
	-0.000455270525591&	0.012524975504886&	1.000000000000000  \\
	
	\end{bmatrix}
	\]
	
	Los errores medios en la \textbf{geometría epipolar} son $\bar{E} =  0.0044$ y $E_{max} = 0.0172$
	
	\subsubsection*{Imágenes 1 a 40}
	
	Para \textbf{cámara 1} se obtienen una media de error de reproyección $\bar{E} = 0.6957$ y $E_{max} = 1.6605$.\\
	Para \textbf{cámara 2} se obtienen una media de error de reproyección $\bar{E} = 0.6270$ y $E_{max} = 1.4962$.\\
	
	La media de error de \textbf{distancia en la reconstrucción} $\bar{E} = 0.1631$ y $E_{max} = 0.6243$.\\
	La matriz fundamental:
	\[
	M=
	\begin{bmatrix}
	0.000000073939914&	0.000001179953828&	-0.000305467200141 \\
	0.000001551893968&	0.000000042322186&	-0.015243660259327 \\
	-0.000455469811366&	0.012514140096993&	1.000000000000000  \\
	
	\end{bmatrix}
	\]
	
	Los errores medios en la \textbf{geometría epipolar} son $\bar{E} = 0.0044$ y $E_{max} = 0.0171$
	
	%%%%%%%%%%%%%%%%%%%%%%%%%%%%	%%%%%%%%%%%%%%%%%%%%%%%%%%%%
	%%%%%%%%%%%%%%%%%%%%%%%%%%%%
	%%%%%%%%%%%%%%%%%%%%%%%%%%%%
	%%%%%%%%%%%%%%%%%%%%%%%%%%%%	%%%%%%%%%%%%%%%%%%%%%%%%%%%%
	
	\subsubsection{Ruido 0 y Puntos Usados 50\%}
	
	\subsubsection*{Imágenes 1 a 10}
	
	Para \textbf{cámara 1} se obtienen una media de error de reproyección $\bar{E} = 5.1730$ y $E_{max} = 6.3470$.\\
	Para \textbf{cámara 2} se obtienen una media de error de reproyección $\bar{E} = 2.8830$ y $E_{max} = 4.5960$.\\
	
	La media de error de \textbf{distancia en la reconstrucción} $\bar{E} = 0.2260$ y $E_{max} = 0.9485$.\\
	La matriz fundamental:
	\[
	M=
	\begin{bmatrix}
	0.000000103480437&	0.000001289011717&	-0.000352355828340 \\
	0.000001463059662&	0.000000114658891&	-0.015197139986200 \\
	-0.000434118625874&	0.012449387223070&	1.000000000000000  \\
	
	\end{bmatrix}
	\]
	
	Los errores medios en la \textbf{geometría epipolar} son $\bar{E} = 0.0043$ y $E_{max} = 0.0157$
	
	\subsubsection*{Imágenes 11 a 20}
	
	Para \textbf{cámara 1} se obtienen una media de error de reproyección $\bar{E} = 5.2190$ y $E_{max} = 11.1360$.\\
	Para \textbf{cámara 2} se obtienen una media de error de reproyección $\bar{E} = 3.2700$ y $E_{max} = 5.5250$.\\
	
	La media de error de \textbf{distancia en la reconstrucción} $\bar{E} = 0.4838$ y $E_{max} = 1.7224$.\\
	La matriz fundamental:
	\[
	M=
	\begin{bmatrix}
	0.000000101531979&	0.000001053330145&	-0.000297428893736 \\
	0.000001640409520&	-0.000000011452609&	-0.015225298660360 \\
	-0.000478085612397&	0.012516538574346&	1.000000000000000  \\
	
	\end{bmatrix}
	\]
	
	Los errores medios en la \textbf{geometría epipolar} son $\bar{E} = 0.0043$ y $E_{max} = 0.0135$
	
	\subsubsection*{Imágenes 21 a 30}
	
	Para \textbf{cámara 1} se obtienen una media de error de reproyección $\bar{E} = 0.5830$ y $E_{max} = 1.7190$.\\
	Para \textbf{cámara 2} se obtienen una media de error de reproyección $\bar{E} = 3.7520$ y $E_{max} = 4.9640$.\\
	
	La media de error de \textbf{distancia en la reconstrucción} $\bar{E} = 0.1877$ y $E_{max} = 0.8394$.\\
	La matriz fundamental:
	\[
	M=
	\begin{bmatrix}
	0.000000117134869&	0.000001086882943&	-0.000290650604694 \\
	0.000001579837482&	0.000000002791701&	-0.015264898850891 \\
	-0.000481127483798&	0.012570785710171&	1.000000000000000  \\
	
	\end{bmatrix}
	\]
	
	Los errores medios en la \textbf{geometría epipolar} son $\bar{E} = 0.0046$ y $E_{max} = 0.0205$
	
	\subsubsection*{Imágenes 31 a 40}
	
	Para \textbf{cámara 1} se obtienen una media de error de reproyección $\bar{E} = 1.5860$ y $E_{max} = 2.9410$.\\
	Para \textbf{cámara 2} se obtienen una media de error de reproyección $\bar{E} = 0.7320$ y $E_{max} = 1.2870$.\\
	
	La media de error de \textbf{distancia en la reconstrucción} $\bar{E} = 0.1341$ y $E_{max} = 0.5086$.\\
	La matriz fundamental:
	\[
	M=
	\begin{bmatrix}
	0.000000069320918&	0.000001133940169&	-0.000291954554812 \\
	0.000001603003049&	0.000000007642188&	-0.015239688700389 \\
	-0.000467309879713&	0.012517530134473&	1.000000000000000 \\
	
	\end{bmatrix}
	\]
	
	Los errores medios en la \textbf{geometría epipolar} son $\bar{E} = 0.0041$ y $E_{max} = 0.0138$
	
	\subsubsection*{Imágenes 1 a 20}
	
	Para \textbf{cámara 1} se obtienen una media de error de reproyección $\bar{E} = 0.7080$ y $E_{max} = 2.0045$.\\
	Para \textbf{cámara 2} se obtienen una media de error de reproyección $\bar{E} = 1.7150$ y $E_{max} = 2.7685$.\\
	
	La media de error de \textbf{distancia en la reconstrucción} $\bar{E} = 0.1604$ y $E_{max} = 0.6294$.\\
	La matriz fundamental:
	\[
	M=
	\begin{bmatrix}
	0.000000095397550&	0.000001230351571&	-0.000333380746614 \\
	0.000001503396463&	0.000000073117846&	-0.015230313991820 \\
	-0.000443685342000&	0.012496661334176&	1.000000000000000  \\
	
	\end{bmatrix}
	\]
	
	Los errores medios en la \textbf{geometría epipolar} son $\bar{E} = 0.0043$ y $E_{max} = 0.0147$
	
	\subsubsection*{Imágenes 1 a 30}
	
	Para \textbf{cámara 1} se obtienen una media de error de reproyección $\bar{E} = 1.0090$ y $E_{max} = 3.1980$.\\
	Para \textbf{cámara 2} se obtienen una media de error de reproyección $\bar{E} = 1.3777$ y $E_{max} = 2.4777$.\\
	
	La media de error de \textbf{distancia en la reconstrucción} $\bar{E} = 0.2378$ y $E_{max} = 0.9220$.\\
	La matriz fundamental:
	\[
	M=
	\begin{bmatrix}
	0.000000095860786&	0.000001178694839&	-0.000313650158006 \\
	0.000001533132550&	0.000000052664553&	-0.015237428671677 \\
	-0.000456790926731&	0.012514201920203&	1.000000000000000  \\
	
	\end{bmatrix}
	\]
	
	Los errores medios en la \textbf{geometría epipolar} son $\bar{E} =  0.0045$ y $E_{max} = 0.0166$
	
	\subsubsection*{Imágenes 1 a 40}
	
	Para \textbf{cámara 1} se obtienen una media de error de reproyección $\bar{E} = 0.6700$ y $E_{max} = 1.6940$.\\
	Para \textbf{cámara 2} se obtienen una media de error de reproyección $\bar{E} = 1.2873$ y $E_{max} = 2.2985$.\\
	
	La media de error de \textbf{distancia en la reconstrucción} $\bar{E} = 0.1460$ y $E_{max} = 0.5575$.\\
	La matriz fundamental:
	\[
	M=
	\begin{bmatrix}
	0.000000090657517&	0.000001155219611&	-0.000305767702595 \\
	0.000001562381437&	0.000000035174222&	-0.015230217361058 \\
	-0.000463107270327&	0.012509197526277&	1.000000000000000  \\
	
	\end{bmatrix}
	\]
	
	Los errores medios en la \textbf{geometría epipolar} son $\bar{E} = 0.0044$ y $E_{max} = 0.0158$
	
	%%%%%%%%%%%%%%%%%%%%%%%%%%%%	%%%%%%%%%%%%%%%%%%%%%%%%%%%%
	%%%%%%%%%%%%%%%%%%%%%%%%%%%%
	%%%%%%%%%%%%%%%%%%%%%%%%%%%%
	%%%%%%%%%%%%%%%%%%%%%%%%%%%%	%%%%%%%%%%%%%%%%%%%%%%%%%%%%
	
	\subsubsection{Ruido 0 y Puntos Usados 25\%}
	
	\subsubsection*{Imágenes 1 a 10}
	
	Para \textbf{cámara 1} se obtienen una media de error de reproyección $\bar{E} = 6.6420$ y $E_{max} = 9.2430$.\\
	Para \textbf{cámara 2} se obtienen una media de error de reproyección $\bar{E} = 12.8840$ y $E_{max} = 40.0590$.\\
	
	La media de error de \textbf{distancia en la reconstrucción} $\bar{E} = 4.6274$ y $E_{max} = 16.3469$.\\
	La matriz fundamental:
	\[
	M=
	\begin{bmatrix}
	0.000000067627702&	0.000001151779685&	-0.000296644327685 \\
	0.000001600037037&	-0.000000025076406&	-0.015306130645137 \\
	-0.000464634721953&	0.012593362485154&	1.000000000000000  \\
	
	\end{bmatrix}
	\]
	
	Los errores medios en la \textbf{geometría epipolar} son $\bar{E} = 0.0040$ y $E_{max} = 0.0113$
	
	\subsubsection*{Imágenes 11 a 20}
	
	Para \textbf{cámara 1} se obtienen una media de error de reproyección $\bar{E} = 4.4980$ y $E_{max} = 7.6080$.\\
	Para \textbf{cámara 2} se obtienen una media de error de reproyección $\bar{E} = 0.3240$ y $E_{max} = 0.8090$.\\
	
	La media de error de \textbf{distancia en la reconstrucción} $\bar{E} = 0.4354$ y $E_{max} = 1.2201$.\\
	La matriz fundamental:
	\[
	M=
	\begin{bmatrix}
	0.000000176606585&	0.000001254554008&	-0.000430007967055 \\
	0.000001454794984&	0.000000012753937&	-0.015370401380612 \\
	-0.000410508897332&	0.012667004375848&	1.000000000000000 \\
	
	\end{bmatrix}
	\]
	
	Los errores medios en la \textbf{geometría epipolar} son $\bar{E} = 0.0041$ y $E_{max} = 0.0121$
	
	\subsubsection*{Imágenes 21 a 30}
	
	Para \textbf{cámara 1} se obtienen una media de error de reproyección $\bar{E} = 5.4330$ y $E_{max} = 9.4850$.\\
	Para \textbf{cámara 2} se obtienen una media de error de reproyección $\bar{E} = 3.3930$ y $E_{max} = 4.2960$.\\
	
	La media de error de \textbf{distancia en la reconstrucción} $\bar{E} = 0.4937$ y $E_{max} = 1.9948$.\\
	La matriz fundamental:
	\[
	M=
	\begin{bmatrix}
	0.000000101602039&	0.000001026308579&	-0.000274983709005 \\
	0.000001682220824&	-0.000000079034171&	-0.015323841704589 \\
	-0.000493826166665&	0.012631537219005&	1.000000000000000 \\
	
	\end{bmatrix}
	\]
	
	Los errores medios en la \textbf{geometría epipolar} son $\bar{E} = 0.0042$ y $E_{max} = 0.0124$
	
	\subsubsection*{Imágenes 31 a 40}
	
	Para \textbf{cámara 1} se obtienen una media de error de reproyección $\bar{E} = 0.3660$ y $E_{max} = 0.8370$.\\
	Para \textbf{cámara 2} se obtienen una media de error de reproyección $\bar{E} = 0.3120$ y $E_{max} = 0.7190$.\\
	
	La media de error de \textbf{distancia en la reconstrucción} $\bar{E} = 0.0784$ y $E_{max} = 0.2921$.\\
	La matriz fundamental:
	\[
	M=
	\begin{bmatrix}
	0.000000093938554&	0.000001004677255&	-0.000271478825857 \\
	0.000001689329775&	-0.000000046011456&	-0.015211557679991 \\
	-0.000496319588187&	0.012512454664390&	1.000000000000000 \\
	
	\end{bmatrix}
	\]
	
	Los errores medios en la \textbf{geometría epipolar} son $\bar{E} = 0.0040$ y $E_{max} = 0.0108$
	
	\subsubsection*{Imágenes 1 a 20}
	
	Para \textbf{cámara 1} se obtienen una media de error de reproyección $\bar{E} = 9.5710$ y $E_{max} = 15.9780$.\\
	Para \textbf{cámara 2} se obtienen una media de error de reproyección $\bar{E} = 0.3115$ y $E_{max} = 0.7870$.\\
	
	La media de error de \textbf{distancia en la reconstrucción} $\bar{E} = 0.6821$ y $E_{max} = 2.0915$.\\
	La matriz fundamental:
	\[
	M=
	\begin{bmatrix}
	0.000000103383383&	0.000001211502408&	-0.000335212660081 \\
	0.000001511239918&	0.000000002979995&	-0.015321279673816 \\
	-0.000445597148994&	0.012617684510306&	1.000000000000000  \\
	
	\end{bmatrix}
	\]
	
	Los errores medios en la \textbf{geometría epipolar} son $\bar{E} = 0.0041$ y $E_{max} = 0.0118$
	
	\subsubsection*{Imágenes 1 a 30}
	
	Para \textbf{cámara 1} se obtienen una media de error de reproyección $\bar{E} = 1.4047$ y $E_{max} = 3.8153$.\\
	Para \textbf{cámara 2} se obtienen una media de error de reproyección $\bar{E} = 0.3367$ y $E_{max} = 0.8280$.\\
	
	La media de error de \textbf{distancia en la reconstrucción} $\bar{E} = 0.1927$ y $E_{max} = 0.7139$.\\
	La matriz fundamental:
	\[
	M=
	\begin{bmatrix}
	0.000000103605792&	0.000001151403656&	-0.000316645683407 \\
	0.000001567490698&	-0.000000021816061&	-0.015302715661521 \\
	-0.000461985689291&	0.012602786523177&	1.000000000000000  \\
	
	\end{bmatrix}
	\]
	
	Los errores medios en la \textbf{geometría epipolar} son $\bar{E} =  0.0042$ y $E_{max} = 0.0121$
	
	\subsubsection*{Imágenes 1 a 40}
	
	Para \textbf{cámara 1} se obtienen una media de error de reproyección $\bar{E} = 0.9755$ y $E_{max} = 1.7570$.\\
	Para \textbf{cámara 2} se obtienen una media de error de reproyección $\bar{E} = 0.3218$ y $E_{max} = 0.7722$.\\
	
	La media de error de \textbf{distancia en la reconstrucción} $\bar{E} = 0.1927$ y $E_{max} = 0.7139$.\\
	La matriz fundamental:
	\[
	M=
	\begin{bmatrix}
	0.000000102230220&	0.000001088199302&	-0.000299389330230 \\
	0.000001619657233&	-0.000000033157892&	-0.015267117114108 \\
	-0.000476423372250&	0.012567756238759&	1.000000000000000  \\
	
	\end{bmatrix}
	\]
	
	Los errores medios en la \textbf{geometría epipolar} son $\bar{E} = 0.0042$ y $E_{max} = 0.0119$
	
	
	\subsection{Conclusión}
	
	Lo más notable es que el \textbf{error medio en la geometría epipolar} nunca ha bajado en todos los ejercicios de $0.004$, pero hablo de la media de errores medios, por lo que habrá errores medios de ciertas muestras que sí están por debajo de $0.004$. Viendo esto se puede asumir que los cálculos de la matriz fundamental no son correctos al encontrarse la media de los errores medios por encima de $0.004$.\\
	
	Otra característica importante es que aumentar el tamaño de la muestra más allá de diez en este caso no reducen los \textbf{errores de reproyección}, aunque a veces lo consiga, otras veces lo aumenta. Tampoco afecta mucho aumentar el ruido o reducir el número de puntos, la inmensa mayoría de los casos de estudios se han mantenido por debajo de $1$ pixel.\\
	
	Como se destaca en algunas tomas de medidas a veces se producen errores enormes, por encima de $100$, esto se debe a errores puntuales de medida, por lo que estos resultados habría que descartarlos.\\
	
	Para el cálculo de las medias de errores se ha exportado el txt a Matlab y de ahí se han seleccionado los datos de interés, \textit{Data\_Selection}, y se han calculado las medias de la siguiente manera:
	\begin{lstlisting}
	a = 0;
	var = Data_Selection;
	for i = 1:size(var,1)
	buff = var{i,1};
	a = a + str2num(buff);
	end
	mean = a / size(var,1)
	\end{lstlisting}
	
\end{document}